\documentclass[[
    ngerman,american,%
    ]{scrartcl}



    \usepackage[ngerman]{babel}
    \usepackage[utf8]{inputenc} 
    \usepackage{csquotes}
    \usepackage{enumitem}
    
    \usepackage[
        bibencoding=utf8, 
        style=alphabetic
    ]{biblatex}
    
    \bibliography{bibliography}
    
    
    \usepackage{amsmath}
    \title{Your Paper's Title \\ \Large{Outline und Themenvorschlag}}
    \author{Author}
    
    \begin{document}
      \maketitle
        \begin{abstract}
            Bitte fügen Sie hier einen ersten Entwurf Ihres Abstracts ein. 
            Bitte gehen Sie hierbei insbesondere auf die folgenden Punkte ein:
            
            \begin{enumerate}
                \item Was ist der Inhalt des Papers?
                \item Warum ist das Thema relevant?
                \item Wie sieht Ihre Antwort/Ergebnisse aus?
                \item Wie belegen Sie, dass Ihre Antwort/Ergebnisse valide sind?
            \end{enumerate}
        \end{abstract}
        
        
        
        \section{Fragestellung/Inhalt - 4 Fragen}
        Bitte beantworten Sie kurz die folgenden 4 Fragen um dazustellen was Sie in Ihrem Paper machen möchten.
        Formatierungsvorschlag:
        
        \begin{description}[style=unboxed]
            \item [Was ist das Problem?] \textit{Ihre Antwort (1-2 Sätze)}
            \item [Warum ist es ein Problem?] \textit{Ihre Antwort (1-2 Sätze)}
            \item [Was ist die Lösung?] \textit{Ihre Antwort (1-2 Sätze)}
            \item [Warum ist es eine Lösung?] \textit{Ihre Antwort (1-2 Sätze)}
        \end{description}
        
        
        \section{Vorläufige Gliederung}
        Bitte fügen Sie hier Ihren aktuellen Vorschlag für eine Gliederung ein. 
        Formatierungsvorschlag:
        
        
        \begin{enumerate}
            \item \textbf{Section 1} \textit{Kurze Beschreibung (2-3 Sätze)}
                    \begin{enumerate}
                        \item \textbf{Subsection 1} \textit{Kurze Beschreibung (2-3 Sätze)}
                        \item \textbf{Subsection 2} \textit{Kurze Beschreibung (2-3 Sätze)}
                    \end{enumerate}
            \item \textbf{Section 2} \textit{Kurze Beschreibung (2-3 Sätze)}
        \end{enumerate}
    
      
            \section{Quellen}
        Bitte fügen Sie hier 10-15 Quellen ein die Sie für Ihr Paper verwenden möchten.
        Formatierungsvorschlag:
        
        \begin{description}
        \item[\cite{gruba_how_2017}] \textit{Kurze Beschreibung der Quelle (1-2 Sätze)}
        \item[\cite{zobel_writing_2015}] \textit{Kurze Beschreibung der Quelle (1-2 Sätze)}
        \end{description}
        
        
      \printbibliography
    \end{document}