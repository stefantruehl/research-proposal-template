\documentclass[
    ngerman,american
    ]{scrartcl}

    % ##########################################
    % # Choose your Language
    % # de = german
    % # en = english
    \newcommand{\lang}{en}
    % ##########################################


    \usepackage[ngerman]{babel}
    \usepackage[utf8]{inputenc} 
    \usepackage{csquotes}
    \usepackage{enumitem}
    \usepackage{ifthen}
    \usepackage{blindtext}
    
    \newcommand{\paperSubTitle}[1]
{
    \ifthenelse{\equal{#1}{en}}{Outline and Topic Proposal}{}
    \ifthenelse{\equal{#1}{de}}{Outline und Themenvorschlag}{}
}

\newcommand{\sectionQuestions}[1]
{
    \ifthenelse{\equal{#1}{en}}{\section{Scope of Work - 4 Questions}}{}
    \ifthenelse{\equal{#1}{de}}{\section{Ziel der Arbeit - 4 Fragen}}{}
}

\newcommand{\sectionQuestionsDescription}[1]
{
    \ifthenelse{\equal{#1}{en}}{In this section the essence of the proposed work is described by answering four key questions. }{}
    \ifthenelse{\equal{#1}{de}}{Im Folgenden wird der Kern der Arbeit beschrieben indem vier Kernfragen beantwortet werden.}{}
}

\newcommand{\sectionInitialTOC}[1]
{
    \ifthenelse{\equal{#1}{en}}{\section{Preliminary Table of Contents}}{}
    \ifthenelse{\equal{#1}{de}}{\section{Vorläufige Gliederung}}{}
}

\newcommand{\sectionInitialTOCDescription}[1]
{
    \ifthenelse{\equal{#1}{en}}{In this section the table of contents for the proposed work is described.}{}
    \ifthenelse{\equal{#1}{de}}{Im Folgenden wird ein Inhaltverzeichnis für die vorgeschlagene Arbeit vorgestellt.}{}
}

\newcommand{\sectionSource}[1]
{
    \ifthenelse{\equal{#1}{en}}{\section{Relevant Related Work}}{}
    \ifthenelse{\equal{#1}{de}}{\section{Relevante verwandte Arbeiten}}{}
}


\newcommand{\sectionSourceDescription}[1]
{
    \ifthenelse{\equal{#1}{en}}{In this section, identified related work is described.}{}
    \ifthenelse{\equal{#1}{de}}{Diese Section stellt verwandte Arbeiten dar und erklärt kurz deren Bedeutung für die vorgeschlagene Arbeit.}{}
}

\newcommand{\questionOne}[1]
{
    \ifthenelse{\equal{#1}{en}}{What is the problem you want to address in your work?}{}
    \ifthenelse{\equal{#1}{de}}{Was ist das Problem, welches Sie in Ihrer Arbeit bearbeiten wollen?}{}
}

\newcommand{\questionTwo}[1]
{
    \ifthenelse{\equal{#1}{en}}{Why is it a problem?}{}
    \ifthenelse{\equal{#1}{de}}{Warum ist es ein Problem?}{}
}

\newcommand{\questionThree}[1]
{
    \ifthenelse{\equal{#1}{en}}{What is the solution you developed in your work?}{}
    \ifthenelse{\equal{#1}{de}}{Was ist die Lösung die sie entwickelt haben?}{}
}

\newcommand{\questionFour}[1]
{
    \ifthenelse{\equal{#1}{en}}{Why is it a solution?}{}
    \ifthenelse{\equal{#1}{de}}{Warum ist es eine Lösung?}{}
}


    \newcommand{\printTrueOrFalse}[1]
    {
        \ifthenelse{\equal{#1}{true}}{TRUE}{}
        \ifthenelse{\equal{#1}{false}}{FALSE}{}
    }



    \usepackage[
        bibencoding=utf8, 
        style=alphabetic
    ]{biblatex}

    \bibliography{bibliography}
    
    
    \usepackage{amsmath}
    \title{[Your Paper's Title] \\  \Large{\paperSubTitle{\lang}}}
    \author{Author}
    
    \begin{document}
      \maketitle
        \begin{abstract}
            \printTrueOrFalse{true}
            \question{\lang}
            Bitte fügen Sie hier einen ersten Entwurf Ihres Abstracts ein. 
            Bitte gehen Sie hierbei insbesondere auf die folgenden Punkte ein:
            
            \begin{enumerate}
                \item Was ist der Inhalt des Papers?
                \item Warum ist das Thema relevant?
                \item Wie sieht Ihre Antwort/Ergebnisse aus?
                \item Wie belegen Sie, dass Ihre Antwort/Ergebnisse valide sind?
            \end{enumerate}
        \end{abstract}
        
        
        \sectionQuestions{\lang}
        Bitte beantworten Sie kurz die folgenden 4 Fragen um dazustellen was Sie in Ihrem Paper machen möchten.
        Formatierungsvorschlag:
        
        \begin{description}[style=unboxed]
            \item [\questionOne{\lang}] 
                % ##########################################
                % # What is the problem? / Was ist das Problem?
                % # 
                include answer here
                % ##########################################

            \item [\questionTwo{\lang}]
                % ##########################################
                % # Why is it a problem? / Warum ist es ein Problem?
                % # 
                include answer here
                % ##########################################

            \item [\questionThree{\lang}]
                % ##########################################
                % #  What is the solution? / Was ist die Lösung?
                % # 
                include answer here
                % ##########################################

            \item [\questionFour{\lang}]
                % ##########################################
                % # Why is it a solution? / Warum ist es eine Lösung?
                % # 
                include answer here
                % ##########################################
        \end{description}
        
        \sectionInitialTOC{\lang}
        Bitte fügen Sie hier Ihren aktuellen Vorschlag für eine Gliederung ein. 
        Formatierungsvorschlag:
        
        
        \begin{enumerate}
            \item \textbf{Section 1} \textit{Kurze Beschreibung (2-3 Sätze)}
                    \begin{enumerate}
                        \item \textbf{Subsection 1} \textit{Kurze Beschreibung (2-3 Sätze)}
                        \item \textbf{Subsection 2} \textit{Kurze Beschreibung (2-3 Sätze)}
                    \end{enumerate}
            \item \textbf{Section 2} \textit{Kurze Beschreibung (2-3 Sätze)}
        \end{enumerate}
    
      
        \sectionSource{\lang}
        %\section{Quellen}
        Bitte fügen Sie hier 10-15 Quellen ein die Sie für Ihr Paper verwenden möchten.
        Formatierungsvorschlag:
        
        \begin{description}
        \item[\cite{gruba_how_2017}] \textit{Kurze Beschreibung der Quelle (1-2 Sätze)}
        \item[\cite{zobel_writing_2015}] \textit{Kurze Beschreibung der Quelle (1-2 Sätze)}
        \end{description}
        
        
      \printbibliography
    \end{document}